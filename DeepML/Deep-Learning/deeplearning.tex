\documentclass[UTF8]{ctexart}
\usepackage{mathtools,wallpaper}
\usepackage{t1enc}
\usepackage{pagecolor}
\usepackage{enumitem}
\usepackage{hyperref}
\usepackage{graphicx}
\usepackage{listings}
\usepackage{amssymb}
\usepackage[dvipsnames]{xcolor}
\usepackage{float}



\begin{document}

\section{Normalization常用归一化公式}

\subsection{Instance Normalization}
实例归一化

对于一个输入特征 $\mathbf{x}$(通常是一个多通道的特征图),\texttt{InstanceNorm} 的计算过程如下:
\begin{equation}
    \hat{x}_{i,c}
    = \frac{x_{i,c} - \mu_c}{\sigma_c + \varepsilon} \times \gamma_c + \beta_c
\end{equation}

其中:
\begin{itemize}
    \item \texttt{x(i,c)} 是通道 $c$ 处的特征值。
    \item $\mu_c$ 是该通道的均值:
    \begin{equation}
        \mu_c = \frac{1}{H \times W}
        \sum_{h=1}^{H} \sum_{w=1}^{W} x_{i,c,h,w}
    \end{equation}
    \item $\sigma_c$ 是该通道的标准差:
    \begin{equation}
        \sigma_c = \sqrt{
            \frac{1}{H \times W}
            \sum_{h=1}^{H} \sum_{w=1}^{W}
            \bigl(x_{i,c,h,w} - \mu_c\bigr)^2
        }
    \end{equation}
    \item $\gamma_c$ 和 $\beta_c$ 是可学习的缩放和平移参数。
    \item $\varepsilon$ 是一个小的常数,用于防止除零错误。
\end{itemize}

\texttt{InstanceNorm} 的归一化计算是针对每个样本的每个通道单独进行的,
与 \texttt{BatchNorm} 不同,它不依赖于 \texttt{mini-batch} 维度。



\end{document}